% https://www.swrit.ru/doc/gost34/34.602-2020.pdf
% Сделано по гост 34.602-2020
{ % чтобы в главном тексте не сбрасывались заголовки
%%%%%%%%%%%%%%%%%%%%% переопределение секций чтобы было 1.1. вместо А.1.1.
\renewcommand*{\thesection}{\arabic{section}}
\titleformat{\section}{\large\bfseries}{\thesection.}{4pt}{}

\renewcommand*{\thesubsection}{\arabic{section}.\arabic{subsection}}
\titleformat{\subsection}{\large\bfseries}{\thesubsection.}{4pt}{}
%%%%%%%%%%%%%%%%%%%%%%%%%%5
{ % для титульного листа
\chapter*{ПРИЛОЖЕНИЕ А Техническое задание на разрабатываемую систему}
\stepcounter{chapter}
\thispagestyle{empty}
\centering

\begin{flushright}
  \MakeUppercase{Утверждено}

  A.B.00001-01 ТЗ 01
\end{flushright}

\vfill

\textbf{\MakeUppercase{Заголовок}}

\textbf{Техническое задание}

\textbf{\textit{Лист утверждения}}

\vbox{
  \parbox{6cm}{
    \begin{sideways}
      \setlength\arrayrulewidth{2pt}
      \begin{tabular}{|c|c|c|c|c|}
        \hline
        Инв. № подл. & Подпись и дата & Взам. инв. № & Инв. № дубл. & Подпись и дата \\
        \hline
                     &&&&\\
        \hline
      \end{tabular}
    \end{sideways}
  }
  \hfill
  \parbox{9.4cm}{
    \begin{flushright}
      \begin{minipage}[t]{0.4\textwidth}
        Руководитель разработки

        \vspace{3mm}
        \makebox[6cm][r]{\hrulefill~Иванов~И.И.}
        \makebox[6cm][r]{<<\rule{7mm}{0.4pt}>>\hrulefill~\the\year}
      \end{minipage}
    \end{flushright}
    \vspace{10mm}
    \begin{flushright}
      \begin{minipage}[t]{0.4\textwidth}
        Исполнитель

        \vspace{3mm}
        \makebox[6cm][r]{\hrulefill~Петров~П.П.}
        \makebox[6cm][r]{<<\rule{7mm}{0.4pt}>>\hrulefill~\the\year}
      \end{minipage}
    \end{flushright}
  }
}
\newpage
}
\section{Общие сведения}
В разделе «Общие сведения» указывают следующее:
\begin{itemize}
    \item полное наименование АС и ее условное обозначение
    \item шифр темы (при наличии);
    \item наименование организации — заказчика АС, наименование организации-разработчика (при наличии сведений о ней);
    \item перечень документов, на основании которых создается АС, кем и когда утверждены эти документы
    \item плановые сроки начала и окончания работ по созданию АС
    \item общие сведения об источниках и порядке финансирования работ
\end{itemize}

П р и м е ч а н и е — К документам, на основании которых или в соответствии с которыми создается АС, могут относиться, например, следующие:
- договорные документы на создание АС;
\begin{itemize}
  \item нормативно-правовые и нормативно-технические документы, регламентирующие создание АС;
  \item техническое задание на создание ранее разрабатывавшейся АС.
\end{itemize}
\section{Цели и назначение создания автоматизированной системы}

\subsection{Цели создания АС}
В подразделе «Цели создания АС» приводят наименования и требуемые значения технических, технологических, производственно-экономических или других показателей объекта автоматизации, которые должны быть достигнуты в результате создания АС, и указывают критерии оценки достижения целей создания АС

\subsection{Назначение АС}
В подразделе «Назначение АС» указывают вид автоматизируемой деятельности (управление, проектирование и т. п.) применительно к объекту автоматизации в целом.

Для сложного объекта автоматизации приводится общий перечень объектов, на которых планируется использовать АС.
\section{Характеристика объекта автоматизации}
В разделе «Характеристика объекта автоматизации» приводят следующую информацию:
\begin{itemize}
  \item основные сведения об объекте автоматизации или ссылки на документы, содержащие такие сведения;
  \item сведения об условиях эксплуатации объекта автоматизации и характеристиках окружающей среды.
\end{itemize}
П р и м е ч а н и е — В разделе приводят основные сведения об объекте автоматизации, позволяющие однозначно его идентифицировать и сформировать правильное представление о масштабах разработки.

\section{Требования к автоматизированной системе}
Состав требований к АС, включаемых в данный раздел ТЗ на АС, устанавливают в зависимости от вида, назначения, специфических особенностей и условий функционирования конкретной автоматизированной системы. В каждом подразделе приводят ссылки на действующие НТД, определяющие требования к автоматизированным системам соответствующего вида.
\subsection{Требования к структуре АС в целом}
В подразделе «Требования к структуре АС в целом» указывают следующее:
\begin{itemize}
  \item перечень подсистем (при их наличии), их назначение и основные характеристики. Дополнительно могут быть приведены требования к числу уровней иерархии и степени централизации АС;
  \item требования к способам и средствам обеспечения информационного взаимодействия компонентов АС;
  \item требования к характеристикам взаимосвязей создаваемой АС со смежными АС, требования к интероперабельности, требования к ее совместимости, в том числе указания о способах обмена информацией;
  \item требования к режимам функционирования АС;
  \item требования по диагностированию АС;
  \item перспективы развития, модернизации АС.
\end{itemize}
\subsection{Требования к функциям (задачам), выполняемым АС}
В подразделе «Требования к функциям (задачам), выполняемым АС», приводят перечень функций (задач), подлежащих автоматизации для АС в целом или для каждой подсистемы (при их наличии). В перечень включаются в том числе функции  (задачи), обеспечивающие взаимодействие частей АС.
Для каждой функции (задачи) должен быть указан результат ее выполнения и, при необходимости, приведены основные характеристики результата. При необходимости дополнительно могут быть указаны следующие данные:
\begin{itemize}
  \item временной регламент реализации каждой функции (задачи);
  \item требования к реализации каждой функции (задачи), к форме представления выходной информации, характеристики необходимой точности и времени выполнения, требования одновременности выполнения группы функций, достоверности выдачи результатов;
  \item перечень и критерии отказов для каждой функции, по которой задаются требования по надежности.
\end{itemize}
\subsection{Требования к видам обеспечения АС}
В подразделе «Требования к видам обеспечения АС» приводят требования к математическому, информационному, лингвистическому, программному, техническому, метрологическому, организационному, методическому и другим видам обеспечения АС.

Для математического обеспечения АС приводят требования к составу, области применения (ограничениям) и способам использования в АС математических методов и моделей, типовых алгоритмов и алгоритмов, подлежащих разработке.

Для информационного обеспечения АС приводят следующие требования:
\begin{itemize}
  \item к составу, структуре и способам организации данных в АС;
  \item к информационному обмену между компонентами АС и со смежными АС;
  \item к информационной совместимости со смежными АС;
  \item по использованию действующих и по разработке новых классификаторов, справочников, форм документов;
  \item по применению систем управления базами данных;
  \item к представлению данных в АС;
  \item к контролю, хранению, обновлению и восстановлению данных.
\end{itemize}
Для лингвистического обеспечения АС приводят следующие требования:
\begin{itemize}
  \item к языкам, используемым в АС, и возможности расширения набора языков (при необходимости);
  \item к способам организации диалога;
  \item к разработке и использованию словарей, тезаурусов;
  \item к описанию синтаксиса формализованного языка.
\end{itemize}
Для программного обеспечения АС приводят следующую информацию:
\begin{itemize}
  \item требования к составу и видам программного обеспечения;
  \item требования к выбору используемого программного обеспечения;
  \item требования к разрабатываемому программному обеспечению;
  \item перечень допустимых покупных программных средств (при наличии).
\end{itemize}
Для технического обеспечения АС приводят следующие требования:
\begin{itemize}
  \item к видам технических средств, в том числе к видам комплексов технических средств, программно-технических комплексов и других комплектующих изделий, допустимых к использованию в АС;
  \item к функциональным, конструктивным и эксплуатационным характеристикам средств технического обеспечения АС.
\end{itemize}
В требованиях к метрологическому обеспечению АС приводят следующую информацию:
\begin{itemize}
  \item количественные значения показателей метрологического обеспечения;
  \item требования к методам (методикам) измерений и измерительного контроля параметров и их характеристик;
  \item требования к средствам измерений и измерительного контроля;
  \item требования к метрологическому обеспечению испытаний АС;
  \item требования к программе метрологического обеспечения АС;
  \item требования к метрологической совместимости технических средств АС;
  \item требования проведения метрологической экспертизы технической документации (при необходимости).
\end{itemize}
Для организационного обеспечения АС приводят следующие требования:
\begin{itemize}
  \item к структуре и функциям подразделений, участвующих в функционировании АС или обеспечивающих эксплуатацию;
  \item к организации функционирования АС и порядку взаимодействия персонала и пользователей АС;
  \item к организации функционирования АС при сбоях, отказах и авариях;
  \item к порядку обеспечения нормативными документами, необходимыми для разработки АС.
\end{itemize}
Для методического обеспечения АС приводят следующую информацию:
\begin{itemize}
  \item перечень применяемых при разработке и функционировании АС нормативно-технических документов (стандартов, нормативов, методик, профилей и т. п.);
  \item порядок и правила обеспечения разработчиков АС нормативно-технической документацией.
\end{itemize}
\subsection{Общие технические требования к АС}
В подразделе «Общие технические требования к АС» указывают следующее:
\begin{itemize}
  \item требования к численности и квалификации персонала и пользователей АС;
  \item требования к показателям назначения;
  \item требования к надежности;
  \item требования по безопасности;
  \item требования к эргономике и технической эстетике;
  \item требования к транспортабельности для подвижных АС;
  \item требования к эксплуатации, техническому обслуживанию, ремонту и хранению компонентов АС;
  \item требования к защите информации от несанкционированного доступа;
  \item требования по сохранности информации при авариях;
  \item требования к защите от влияния внешних воздействий;
  \item требования к патентной чистоте и патентоспособности;
  \item требования по стандартизации и унификации;
  \item дополнительные требования.
\end{itemize}
В требованиях к численности и квалификации персонала и пользователей АС приводят следующее:
\begin{itemize}
  \item требования к численности персонала и пользователей АС;
  \item требования к квалификации персонала и пользователей АС, порядку их подготовки и контроля знаний и навыков;
  \item требуемый режим работы персонала и пользователей АС.
\end{itemize}
В требованиях к показателям назначения АС приводят значения параметров, характеризующих степень соответствия АС ее назначению (при их наличии).

В требования к надежности включают:
\begin{itemize}
  \item состав и количественные значения показателей надежности для АС в целом или ее подсистем (составных частей);
  \item перечень аварийных ситуаций, по которым должны быть регламентированы требования к надежности, и значения соответствующих показателей;
  \item требования к надежности технических средств и программного обеспечения;
  \item требования к методам оценки и контроля показателей надежности на разных стадиях создания АС в соответствии с действующими нормативно-техническими документами.
\end{itemize}
В требования по безопасности включают требования по обеспечению безопасности при монтаже, наладке, эксплуатации, обслуживании и ремонте технических средств АС (защита от воздействий электрического тока, электромагнитных полей и т. п.), по допустимым уровням вибрационных и шумовых нагрузок, а также по обеспечению экологической безопасности.

В требования к эргономике и технической эстетике включают следующие требования:
\begin{itemize}
  \item эргономические требования к организации и средствам деятельности персонала и пользователей АС, в том числе к средствам отображения информации и организации рабочего места;
  \item требования к технической эстетике, определяющие композиционную целостность, информационную выразительность, рациональность формы и культуру производственного исполнения создаваемого изделия, в том числе реализации человеко-машинного интерфейса.
\end{itemize}
В требования к транспортабельности для подвижных АС включают конструктивные требования, обеспечивающие транспортабельность технических средств АС, а также требования к транспортным средствам, включая условия транспортирования, возможность перевозки в готовом к функционированию состоянии, необходимость защиты элементов АС от внешних воздействующих факторов при транспортировании, а также требования безопасности перевозки.

В требования к эксплуатации, техническому обслуживанию, ремонту и хранению компонентов АС включают:
\begin{itemize}
  \item условия и регламент (режим) эксплуатации, которые должны обеспечивать использование технических средств (ТС) и программно-технических средств (ПТС) АС с заданными показателями;
  \item требования к видам, периодичности и объему технического обслуживания, контролю технического состояния и ремонта или допустимость работы без обслуживания;
  \item предварительные требования к допустимым площадям для размещения персонала и технических средств АС, к параметрам сетей энергоснабжения, вентиляции, охлаждения и т. п.;
  \item требования к составу, размещению и условиям хранения комплекта запасных частей, инструментов и принадлежностей, а также к нормам расхода запасных частей;
  \item требования к регламенту обслуживания.
\end{itemize}
В требования к защите информации от несанкционированного доступа включают требования, установленные в НТД, действующей в отрасли (ведомстве) заказчика.

В требованиях по сохранности информации приводят перечень событий: аварий, отказов технических средств (в том числе — потеря питания) и т. п., при которых должна быть обеспечена сохранность информации в АС.

В требованиях к защите от внешних воздействий приводят:
\begin{itemize}
  \item требования к радиоэлектронной защите средств АС;
  \item требования по стойкости, устойчивости и прочности к внешним воздействиям (среде применения)
\end{itemize}

В требованиях к патентной чистоте и патентоспособности указывают требования по патентной чистоте и патентоспособности АС и ее частей, включая требования по проведению патентных исследований.

В требования к стандартизации и унификации включают показатели, устанавливающие следующее:
\begin{itemize}
  \item требуемую степень использования стандартных, унифицированных методов реализации функций (задач) АС, поставляемых программных средств, типовых математических методов и моделей, типовых проектных решений, унифицированных форм документов, общероссийских классификаторов и классификаторов других категорий в соответствии с областью их применения;
  \item требования к использованию типовых автоматизированных рабочих мест, компонентов и комплексов
\end{itemize}
В дополнительные требования включают:
\begin{itemize}
  \item требования к оснащению АС учебно-тренировочными средствами и документацией на них;
  \item требования к сервисной аппаратуре, стендам для проверки элементов АС;
  \item требования к АС, связанные с особыми условиями эксплуатации;
  \item специальные требования по усмотрению разработчика или заказчика АС.
\end{itemize}
\section{Состав и содержание работ по созданию автоматизированной системы}
Раздел «Состав и содержание работ по созданию автоматизированной системы» должен содержать перечень этапов работ по созданию АС и сроки их выполнения.

\section{Порядок разработки автоматизированной системы}
В разделе «Порядок разработки автоматизированной системы» приводят следующее:
\begin{itemize}
  \item порядок организации разработки АС;
  \item перечень документов и исходных данных для разработки АС;
  \item перечень документов, предъявляемых по окончании соответствующих этапов работ;
  \item порядок проведения экспертизы технической документации;
  \item перечень макетов (при необходимости), порядок их разработки, изготовления, испытаний, необходимость разработки на них документации, программы и методик испытаний;
  \item порядок разработки, согласования и утверждения плана совместных работ по разработке АС;
  \item порядок разработки, согласования и утверждения программы работ по стандартизации;
  \item требования к гарантийным обязательствам разработчика;
  \item порядок проведения технико-экономической оценки разработки АС;
  \item порядок разработки, согласования и утверждения программы метрологического обеспечения, программы обеспечения надежности, программы эргономического обеспечения.
\end{itemize}

\section{Порядок контроля и приемки автоматизированной системы}
В разделе «Порядок контроля и приемки автоматизированной системы» указывают следующую информацию:
\begin{itemize}
  \item виды, состав и методы испытаний АС и ее составных частей;
  \item общие требования к приемке работ, порядок согласования и утверждения приемочной документации;
  \item статус приемочной комиссии (государственная, межведомственная, ведомственная и др.).
\end{itemize}
П р и м е ч а н и е — Порядок согласования и утверждения приемочной документации, а также статус приемочной комиссии указываются при необходимости.

\section{Требования к составу и содержанию работ по подготовке объекта автоматизации к вводу автоматизированной системы в действие}
В разделе «Требования к составу и содержанию работ по подготовке объекта автоматизации к вводу автоматизированной системы в действие» приводят перечень мероприятий, которые необходимо осуществить при подготовке объекта автоматизации к вводу АС в действие.

В перечень мероприятий включают следующее:
\begin{itemize}
  \item создание условий функционирования объекта автоматизации, при которых гарантируется соответствие создаваемой АС требованиям, содержащимся в ТЗ на АС;
  \item проведение необходимых организационно-штатных мероприятий;
  \item порядок обучения персонала и пользователей АС.
\end{itemize}

\section{Требования к документированию}
В разделе «Требования к документированию» приводят следующую информацию:
\begin{itemize}
  \item перечень подлежащих разработке документов;
  \item вид представления и количество документов;
  \item требования по использованию ЕСКД и ЕСПД при разработке документов.
\end{itemize}
При отсутствии государственных стандартов, определяющих требования к документированию
элементов АС, дополнительно включают требования к составу и содержанию таких документов.

\section{Источники разработки}
В разделе «Источники разработки» должны быть перечислены документы и информационные материалы (технико-экономическое обоснование, отчеты о законченных научно-исследовательских работах, информационные материалы на отечественные, зарубежные системы-аналоги и др.), на основании которых разрабатывалось ТЗ и которые должны быть использованы при создании АС.
}
