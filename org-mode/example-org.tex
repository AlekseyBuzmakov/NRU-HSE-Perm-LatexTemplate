% Created 2022-12-03 Сб 16:34
% Intended LaTeX compiler: pdflatex
\documentclass[PI,KR]{HSEUniversity}
         \usepackage{array,tabularx,tabulary,booktabs,longtable,multirow}
         \Year{\the\year{}}
         \supervisor{к.т.н., доцент кафедры Информационных технологий в бизнесе НИУ ВШЭ-Пермь}{А. В. Бузмаков}
                  \Abstract{ После титульного листа размещается краткая (до 0,5 стр.) аннотация, предназначенная для реферативных изданий (например, журналы ВИНИТИ) и библиотечных информационных систем. В ней перечисляются автор, наименование работы; о чем она написана и для кого; количество страниц, иллюстраций, год, издательство (в данном случае – кафедра). Пример аннотации можно увидеть в любой книге на обороте титульного листа. Аннотации работ используются при формировании каталога работ, выполненных на кафедре. Текст аннотации оформляется в соответствии с правилами оформления основного текста работы.}


\usepackage[utf8]{inputenc}
\usepackage[T1]{fontenc}
\usepackage{graphicx}
\usepackage{longtable}
\usepackage{wrapfig}
\usepackage{rotating}
\usepackage[normalem]{ulem}
\usepackage{amsmath}
\usepackage{amssymb}
\usepackage{capt-of}
\usepackage{hyperref}
\author{Семен Семеныч Сидоров}
\date{\today}
\title{Краткий шаблон, демонстрирующий использование различных команд \LaTeX}
\hypersetup{
 pdfauthor={Семен Семеныч Сидоров},
 pdftitle={Краткий шаблон, демонстрирующий использование различных команд \LaTeX},
 pdfkeywords={},
 pdfsubject={},
 pdfcreator={Emacs 28.2 (Org mode 9.6)}, 
 pdflang={English}}
\usepackage{biblatex}
\addbibresource{/home/samoed/Desktop/NRU-HSE-Perm-LatexTemplate/org-mode/library.bib}
\begin{document}

\maketitle

\chapter*{Введение}
\label{sec:org98c3ad0}
Слова в тексте могут быть выделены, например: \textbf{Введение} представляет собой наиболее ответственную часть любой работы. Также покажем как пользоваться << кавычками>>. И << Кавычками ``внутри'' кавычек>>.

Далее будет продемонстрирована работа основных команд \(\LaTeX{}\).
\chapter{Написание работы}
\label{sec:orgb70bb72}
\section{Пример нумерованных списков}
\label{sec:org68f8fc0}
Традиционно во введении:
\begin{enumerate}
\item обосновывается \textbf{актуальность} выбранной
\item формулируется \textbf{цель работы} и \textbf{содержание поставленных задач}, излагается их суть;
\item описываются \textbf{объект} и \textbf{предмет} исследования;
\item освещается \textbf{степень разработанности} данной проблемы;
\item указывается направление и \textbf{избранный метод (методы)} исследования, подходы к решению поставленных задач или реализации новой разработки;
\item указывается, что нового вносится автором в предмет исследования, отмечается \textbf{теоретическая значимость} и \textbf{прикладная ценность} планируемых результатов;
\item формулируются \textbf{основные положения, которые автор выносит на защиту}.
\end{enumerate}
\section{Пример маркированных списков}
\label{sec:org1a9c5c1}
Традиционно во введении:
\begin{itemize}
\item обосновывается \textbf{актуальность} выбранной
\item формулируется \textbf{цель работы} и \textbf{содержание поставленных задач}, излагается их суть;
\item описываются \textbf{объект} и \textbf{предмет} исследования;
\item освещается \textbf{степень разработанности} данной проблемы;
\item указывается направление и \textbf{избранный метод (методы)} исследования, подходы к решению поставленных задач или реализации новой разработки;
\item указывается, что нового вносится автором в предмет исследования, отмечается \textbf{теоретическая значимость} и \textbf{прикладная ценность} планируемых результатов;
\item формулируются \textbf{основные положения, которые автор выносит на защиту}
\end{itemize}

\section{Заголовки разного уровня}
\label{sec:orgcb6242c}
Могут быть еще подразделы
\subsection{Подраздел}
\label{sec:org158537a}
И под-подразделы
\subsubsection{Под-подраздел}
\label{sec:org60f82a6}
Текст

\section{Оформление таблиц}
\label{sec:org822a6bc}
В качестве примера таблицы см. табл. \ref{tbl:example}.
\begin{table}[htbp]
\caption{\label{tbl:example}Пример таблицы}
\centering
\begin{tabular}{c|cc}
\hline
\hline
Заголовок 1 & Заголовок 2 & Заголовок 3\\\empty
\hline
1 & 2 & 3\\\empty
4 & 5 & 6\\\empty
\hline
\hline
\end{tabular}
\end{table}

\section{Оформление формул}
\label{sec:orgfc54d7c}
Для оформления формул используются стандартные средства \(\LaTeX{}\) примеры inline (внутри-строчных формул): \(A^{i}_{k}\), \(A_{k}^{i}\), \(1+2+\dots+n\), \(x_{1}, x_{2}, \dots, x_{n}\).

Формулы, за исключением формул, помещаемых в приложении, должны нумероваться сквозной нумерацией арабскими цифрами, которые записывают на уровне формулы справа в круглых скобках, например:
\begin{equation}
\label{eq:formula-1}
   X^{*}=\frac{P_{s}-P_{p}/n+\overline{Div}}{(P_{s}+P_{p})/2},
\end{equation}

где \(r\) -- доходность от операций с акцией;

\(P_{s}\) -- цена продажи акции;

\(\overline{Div}\) -- средний дивиденд за n лет (определяется как среднее арифметическое);
\(n\) -- число лет с момента покупки до момента продажи акции.

Пояснение каждого символа следует давать с новой строки в той же последовательности, что и в формуле. Первая строка пояснение должна начинаться со слова где без двоеточия после него.

Ссылки в тексте на порядковые номера формул дают в скобках, например, <<\(\dots{}\) в формуле \ref{eq:formula-1} \(\dots{}\) >>.
\section{Оформление иллюстраций}
\label{sec:org5585a85}
На рис. \ref{fig:smile} --- пример картинки, которая распалагается в том месте, в котором она была расположена.

\begin{figure}[h]
\centering
\includegraphics[width=0.4\textwidth]{img/fig.png}
\caption{\label{fig:smile}Пример картинки (в месте расположения)}
\end{figure}
\section{Оформление списка литературы}
\label{sec:org6ef0bcb}
Ссылки на источник в \(\LaTeX{}\) даются командой \texttt{\textbackslash{}cite} вне зависимости от типа источника. Информация об источниках должна быть размещена в \texttt{bibtex} файле, в данном случае в файле \texttt{library.bib}. Описание источника начинается с указания его типа \texttt{@article}, \texttt{@book}, \texttt{@inpoceeding}, \texttt{@online} и\textasciitilde{}др., далее идёт описание специфичных полей для этого типа источника. Сам файле указывается в самом начале \(\LaTeX{}\) файла командой \texttt{\textbackslash{}bibliography}. Для управления списком литературы рекомендуется использовать специализированные системы, например, \texttt{Mendeley}.

Команда \texttt{\textbackslash{}cite} позволяет ссылаться на все типы источников по их имени: напимер, ссылка на книгу\textasciitilde{}\autocite{BookExample}, или на сайт\autocite{HSEDocuments2}\textasciitilde{}. Также можно ссылаться сразу на несколько источников \autocites{BookExample}[][]{ConferencePaperExample}[][]{ArticleExample}. Обратите внимание, что в этом случае нельзя добавлять пробелы между именами источников.

Список литературы вставляется специальной командой \texttt{\textbackslash{}putbibliography}, далее \(\LaTeX{}\) берёт на себя оформление списка литературы.

\putbibliography
\appendix
\chapter{ТЗ}
\label{sec:org0b73e48}
text text
\end{document}