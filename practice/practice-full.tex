\documentclass[PI,VKR]{HSEUniversityPractice}
% Возможные опции: KR или VKR; PI или BI

\title{Выжимка из правил работы над курсовой работой}
\author{Семён Семёныч Сидоров}
\supervisor{к.т.н.}{доцент}{И.И.~Иванов}
%\supervisor{к.т.н.}{доцент кафедры Информационных технологий в бизнесе НИУ ВШЭ-Пермь}{И.И.~Иванов}
% \company{старший программист}{П.П.~Петров}
\Group{ПИ-19-1}
\Year{\the\year}
\Abstract{
	После титульного листа размещается краткая (до 0,5 стр.) аннотация, предназначенная для реферативных изданий (например, журналы ВИНИТИ) и библиотечных информационных систем. В ней перечисляются автор, наименование работы; о чем она написана и для кого; количество страниц, иллюстраций, год, издательство (в данном случае – кафедра). Пример аннотации можно увидеть в любой книге на обороте титульного листа. Аннотации работ используются при формировании каталога работ, выполненных на кафедре. Текст аннотации оформляется в соответствии с правилами оформления основного текста работы.
}

% Ссылка на файл с описание библиографии
\bibliography{../library.bib}

%%%%%%%%%%%%%%%%%%%%%%%%%%%%%%%%
%%% ТЕКСТ РАБОТЫ %%%%%%%%%%%%%%%
\begin{document}

% Обязательные элементы оформления: заголовочный слайд, аннотация, оглавление
\maketitle

\chapter*{Введение}

Данный текст составлен на основании требований к ВКР по направлению <<Программная инженерия>>. Точные требования могут быть найдены на сайте соответсвующей образовательной программы. Здесь приведены лишь некоторые выжимки из вышеозначенного документа. Основной целью этого файла является демонстрация основных принципов работы с файлами \LaTeX{}.

\emph{Введение} представляет собой наиболее ответственную часть любой работы, поскольку содержит в сжатой форме все основные положения, изложению, обоснованию и реализации которых посвящена работа.

Традиционно во введении:
\begin{enumerate}
	\item  обосновывается \emph{актуальность} выбранной темы;
	\item  формулируется \emph{цель работы} и \emph{содержание поставленных задач}, излагается их суть;
	\item  описываются \emph{объект} и \emph{предмет} исследования;
	\item  освещается \emph{степень разработанности} данной проблемы;
	\item  указывается направление и \emph{избранный метод (методы)} исследования, подходы к решению поставленных задач или реализации новой разработки;
	\item  указывается, что нового вносится автором в предмет исследования, отмечается \emph{теоретическая значимость} и \emph{прикладная ценность} планируемых результатов;
	\item  формулируются \emph{основные положения, которые автор выносит на защиту}.
\end{enumerate}

Во введение можно также включить краткое содержание работы по главам, описать структурные особенности дальнейшего изложения материала и обосновать логику его построения. Весь порядок изложения материала работы должен быть направлен на достижение поставленной цели. Логичность изложения работы достигается только тогда, когда каждая глава имеет определенное целевое назначение и является базой для последующей главы.

Обоснование \emph{актуальности темы} должно содержать объяснение того, почему к данной теме целесообразно обратиться именно сейчас, какова научная и практическая необходимость, в каком состоянии находятся современные научные представления о предмете исследования и практические разработки в данной области.

Рассмотрение \emph{степени разработанности} проблемы включает перечисление существующих подходов к решению актуальных задач, наиболее значимых результатов отечественных и зарубежных ученых, занимавшихся данной проблемой, имеющихся в данной области разработок; а также указание того, какие вопросы остаются недостаточно освещенными, какие недостатки и ограничения присущи выполненным ранее работам. (Названия основных трудов отечественных и зарубежных исследователей, относящихся к теме работы, существующих программных продуктов и т.д. можно указать в сносках или привести в библиографическом списке.).

Обосновать выбор темы можно, например, недостаточной ее исследованностью или созданием новых условий для решения указанных проблем, в которых имеющиеся решения оказываются неэффективными (появление новых технологий и т.п.).

Изложение материала должно продемонстрировать, что автор хорошо ориентируется в поставленной проблеме, овладел методами научной работы с библиографическим материалом, может верно оценить вклад предшественников в решение данной проблемы. Важно дать обоснованную критическую оценку выполненных ранее значимых работ, отметить их главные достоинства и недостатки. 

После рассмотрения степени научной разработанности проблемы формулируется место представляемой автором работы в исследовании поставленной проблемы, т.е. \emph{цель} работы и ее \emph{задачи} (<<стратегия>> и <<тактика>>).

Проблемная ситуация всегда связана с некоторым \emph{объектом}, который избирается для изучения. \emph{Предмет} исследования – логическое описание объекта. В объекте выделяется та его часть, которая служит предметом исследования.

\emph{Цель} работы раскрывает ее тему. Перечисление задач, поставленных в работе для достижения сформулированной цели, фактически задает план и внутреннюю логику текста всей работы. 

Автор должен дать объективную оценку собственного вклада в решение поставленной проблемы, степени научной новизны выполненной работы и ее практической ценности. Если у автора возникло ощущение, что до него никто не обращался к данной теме, лучше вернуться к анализу имеющейся литературы, проконсультироваться с руководителем, после чего принять решение, какие положения можно выносить на защиту. 

Следует отметить, что введение изучается всеми заинтересованными лицами: от руководителя и рецензента до членов государственных комиссий, 
и именно по введению составляется первое представление о работе и ее уровне.

Приступая к написанию работы, нельзя сразу писать ее начало --- введение.
В частности, то, какие основные положения выносятся на защиту и их оценка, может окончательно оформиться только на последнем этапе работы. После написания основной части текста работы, возможно, может понадобиться вернуться к оформлению введения. 

\chapter{Основная часть работы}

Основная часть работы должна составлять не менее 70\% ее полного объема. Она делится на главы и параграфы в соответствии с логической структурой изложения. В работе может быть 2‑3 главы или более. Каждая глава состоит не менее чем из двух параграфов.

Логическая структура работы может быть представлена в виде плана, отражающего содержание работы как логического целого, построенного в виде развернутого доказательства положений, обоснования решений, которые выносятся на защиту.

Деление работы на главы и параграфы должно служить логике раскрытия темы. Пункты плана должны структурно полностью раскрывать тему, но не следует вводить в план разделы, содержательно выходящие за рамки темы или связанные с ней лишь косвенно.

Главы --- это основные структурные единицы текста работы. Название каждой из них нужно сформулировать так, чтобы оно не оказалось шире темы всей работы, так как глава представляет только один из аспектов темы, одну из сторон в решении поставленных задач и название должно отражать эту подчиненность.
Каждая глава должна заканчиваться выводами и постановкой задачи для изложения материала следующих глав.

Первая глава, как правило, содержит обстоятельный обзор научной литературы и существующих решений за последние годы, известных исследований и разработок, их анализ, а также материалы, показывающие, что необходимо выполнить для решения поставленных в работе задач и как это сделать наиболее рационально. В этой главе (в отдельных параграфах) дается краткий критический анализ выполненных ранее работ, где необходимо назвать те вопросы, которые остались нерешенными, а также указать, какие из полученных ранее результатов могут быть использованы при решении задач, поставленных в представляемой автором работе. Полная, детальная (в отличие от <<введения>>) математическая постановка задачи может содержаться в первой или начале второй главы.

Вторая глава может быть посвящена изложению теоретического обоснования решаемой задачи. Назначение этой главы – дать теоретический материал по вопросам, рассматриваемым в работе, с точки зрения его применения для достижения поставленной цели, найти необходимую теоретическую основу для решения поставленных задач.

Третья глава, как правило, содержит описание методов исследований, используемых технологий, инструментальных средств. Ее назначение – конкретизировать обобщенное теоретическое решение задачи, выбранный подход к ее решению.

Четвертая глава может содержать решение конкретной задачи со всеми обоснованными и разработанными методиками, моделями, условиями и т.п. Здесь приводится структура и описание разработанных автором алгоритмов, методологии, программного обеспечения, т.е. всего, что является результатом всей работы.

Обсуждению и оценке полученных и представленных в данной главе результатов следует посвятить отдельный параграф. Оценка результатов работы должна быть качественной и количественной с представлением графической информации, табличных данных, диаграмм. Сравнение с известными решениями следует проводить по всем аспектам, в том числе и по эффективности. Следует указать на возможность обобщений, дальнейшего развития методов и идей, использования результатов работы в смежных областях.

В заключении подводятся итоги работы. Формулируются основные выводы по результатам исследований. Приводятся сведения об апробации, об опубликовании основного содержания работы (если имеются публикации), ее результатов, выводов. Приводятся сведения о защищенности технических решений авторскими свидетельствами (патентами). Указывается, где внедрены результаты работы, и где еще они могут быть использованы.

Заключение имеет особую важность, поскольку именно здесь в завершенной форме должны быть представлены итоговые результаты работы. В заключении объединяются отдельные результаты по теме и совокупный итог работы в целом. Здесь необходимо соотнести полученные выводы с целями и задачами, поставленными во введении, соединить в единое целое сделанные в предшествующих главах выводы, оценить успешность собственной работы.

Целесообразно построить текст заключения как перечень выводов, разбив его на пункты, каждый из которых – выделение и обоснование одного конкретного вывода. Если работа наряду с теоретическими результатами имеет и практическую значимость, это также должно быть отмечено в заключении.
Кроме того, следует оценить открывающуюся на основе результатов выполненной работы перспективу дальнейших исследований по данной теме, очертить встающие в этой связи новые задачи, охарактеризовать дополнительные (<<не запланированные>> при первоначальной постановке задачи) результаты и идеи, а также оценить возможные перспективы их развития и использования.

Если в тексте работы использованы свои (не общепринятые) обозначения и сокращения, их список можно привести на отдельной странице, следующей сразу же за заключением. Если сокращения, условные обозначения, символы, единицы и термины повторяются в отчете менее трех раз, отдельный список не составляют, а расшифровку дают непосредственно в тексте работы при первом их упоминании.

Библиографический список представляет собой перечень литературных источников, использованных автором в ходе работы над темой. Список следует за заключением.
Каждый включенный в такой список литературный источник необходимо отразить в рукописи работы. Не стоит включать в библиографический список те источники, на которые нет ссылок в тексте ВКР и которые не были использованы при выполнении работы, а также энциклопедии, справочники, научно-популярные книги, газеты и т.п. Если есть необходимость в использовании таких изданий, то лучше сделать ссылки на них с помощью подстрочных сносок.

Библиографический список оформляется в соответствии с правилами, описанными ниже.

Вспомогательные или дополнительные материалы справочного характера, которые загромождают текст основной части работы, помещают в приложении.
По содержанию и оформлению приложения могут быть очень разнообразны: копии подлинных документов, выдержки из отчетов, отдельные положения из инструкций и правил и т.п. Приложения могут содержать тексты программ и результаты решения задач с их помощью, таблицы, рисунки (графики, диаграммы, схемы и т.д.), выводы формул, но не текст, вынесенный с целью сокращения объема работы.

\section{Типичные ошибки в изложении материала}

Новые тексты часто создаются на основе некоторого образца. К сожалению, в последние годы издается мало научной литературы для программистов. Поэтому основным образцом для подражания становится руководство пользователя по какому-либо программному продукту. Это наихудший образец. Руководство пользователя декларативно, в нем полностью отсутствуют <<аналитичность>> и математика, не излагаются конструктивные решения (как устроена программа <<внутри>>).

Иногда отсутствует математическая постановка задачи. У студента создается впечатление, что он не может сделать математическую постановку для своей задачи, так как ни с какими дифференциальными уравнениями и т.п. его работа не связана. Не следует забывать, что программист работает со сложными структурами данных, имеющими внутренние логические связи; программы (потоки управления) имеют сложную структуру; программы взаимодействуют с внешними процессами и с множеством пользователей, описываемыми своими закономерностями. Поэтому арсенал дискретной математики, математической логики, теории графов, теории автоматов, теории кодирования, а также теории вероятностей и других вполне математических дисциплин – в распоряжении студента.

\chapter{Требования к оформлению работы}

Оффициальные требования могут быть найдены на сайте соответствующей образовательный программы. Основные требования к оформлению текста данный пакет берёт на себ. Здесь приводятся лишь некоторые выжимки из этих документов.

Каждый абзац должен содержать законченную мысль и состоять, как правило, из 4‑5 предложений. 
Слишком <<крупный>> абзац затрудняет восприятие смысла и свидетельствует о неумении четко излагать мысль.
В работах должны применяться научно-технические термины, обозначения и определения, установленные соответствующими стандартами, а при их отсутствии --- общепринятые в научно-технической литературе. Если в работе используется специфическая терминология, то в конце работы (перед списком литературы) должен быть перечень принятых терминов с соответствующими разъяснениями (глоссарий). Перечень включают в содержание работы.

В тексте работы \emph{не допускается}:
\begin{itemize}
	\item применять обороты разговорной речи, техницизмы, профессионализмы;
	\item использовать для одного и того же понятия различные научно-технические термины, близкие по смыслу (синонимы), а также иностранные слова и термины при наличии равнозначных слов и терминов в русском языке;
	\item применять произвольные словообразования, сокращения слов, кроме установленных правилами русской орфографии, соответствующими государственными стандартами, а также приведенных в самой работе;
	\item сокращать обозначения единиц физических величин, если они употребляются без цифр, за исключением единиц физических величин в заголовках и боковиках таблиц в расшифровках буквенных обозначений, входящих в формулы и рисунки.
\end{itemize}

Кроме того, в тексте работы, за исключением формул, таблиц и рисунков, не допускается применять математический знак минус перед отрицательными значениями величин (следует писать слово <<минус>>) применять знак <<$emptyset$>> для обозначения диаметра (следует писать слово <<диаметр>>); применять без числовых значений математические знаки, например $>$ (больше), $<$ (меньше), $=$ (равно), $\geq$ (больше или равно) и т.п.

Наименования команд, режимов, сигналов и т.п. в тексте следует выделять кавычками и шрифтом, например, <<\texttt{Ctrl + Alt + Del}>> или <<\texttt{Файл $\rightarrow$ Отправить сообщение\dots}>>.

При необходимости применения условных обозначений, изображений или знаков, не установленных действующими стандартами, их следует пояснять в тексте или в перечне обозначений.

В документе следует применять стандартизованные единицы физических величин, их наименования и обозначения в соответствии с ГОСТ 8.417-20021\footnote{ГОСТ 8.417-2002. Единицы величин. М., 2002. 24 с. (Государственная система обеспечения единства измерений.)}.

Правила технического редактирования текста запрещают размещение в разных строках чисел и их наименований (например: 1991~год, 10~пунктов и т.п.). Для предотвращения нежелательных переносов слов на следующие строки между числом и его наименованием следует вставлять не обычный пробел, а неразрывный (фиксированный) пробел. Запрещено отрывать инициалы от фамилий, предлоги, начинающие предложения, от следующих за ними слов, разрывать сокращенные выражения (<<т.е.>>, <<и др.>>) и т.д. В \LaTeX{} для этого используется символ \textasciitilde{} (например: \texttt{1991\textasciitilde{}год} будет выгледеть в документе как <<1991~год>>).

\section{Оформление таблиц} 
%
Строки заголовков должны быть набраны по центру ячеек. В каждой таблице следует указывать единицы измерения показателей и период времени, к которому относятся данные. Если единица измерения в таблице является общей для всех числовых табличных данных, то ее приводят в заголовке таблицы после ее названия.
Данные в ячейках таблиц должны быть единообразно выровнены по всей высоте столбца. Если для числовых данных есть итоговая строка, то обязательно выравнивание числовых данных по разрядам. Если числовые данные представляют собой интервалы (пары чисел, разделенных тире), они должны выравниваться по тире. Если встречаются пятизначные числа и более, то цифры разбиваются на классы (группы) с помощью неразрывных пробелов (при использовании обычного пробела вычисления в таблицах будут производиться с ошибками). Если существуют повторения в смежных ячейках, повторяющиеся данные могут быть заменены кавычками. Повторяющиеся текстовые данные допустимо при повторении заменить словами «То же».

На все размещенные в работе таблицы должны быть ссылки в ее тексте. Ссылка на таблицу задается в формате «табл. C.N», где С – номер раздела (главы), а N – номер таблицы в этом разделе. Сокращение «см.» используется, если таблица и ссылка расположены на разных страницах, например «см. табл. 1.3».

Таблицу в документе желательно размещать после ссылки на нее, в пределах разворота, на котором имеется ссылка. Если таблица имеет большой объем, то она может быть вынесена в приложение.

В качестве примера таблицы см.~табл.~\ref{tbl:example-table} .

\begin{TABLE}[h]{Пример таблицы\label{tbl:example-table}}
	\begin{tabular}{c|cc}
		\hline\hline % Двойная линия сверху таблицы
		Заголовок 1 & Заголовок 2 & Заголовок 3 \\ 
		\hline % Одинарная линия между заголовками и телом таблицы
		1 & 2 & 3 \\
		4 & 5 & 6 \\
		\hline\hline % Двойная линия снизу таблицы
	\end{tabular}
\end{TABLE}

\section{Оформление формул}

Для оформления формул используются стандартные средства \LaTeX{}. Примеры <<inline>> (внутри-строчных формул): $A^i_k$, $A_k^i$, $1+2+\cdots+n$, $x_1, x_2, \dots, x_n$.

Формулы, за исключением формул, помещаемых в приложении, должны нумероваться сквозной нумерацией арабскими цифрами, которые записывают на уровне формулы справа в круглых скобках, например:
\begin{equation}
	X^* = \frac{r_p}{\sqrt{(M-m_0\cdot I)V^{-1}(M-m_0\cdot I)}}\cdot V^{-1}(M-M_0\cdot I)
	\label{eq:formula-1}
\end{equation}

Пояснения символов и числовых коэффициентов, входящих в формулу, если они не пояснены ранее в тексте, должны быть приведены непосредственно под формулой:
\begin{equation}
	r = \frac{(P_s-P_p)/n + \overline{Div}}{(P_s + P_p)/2},
	\label{eq:formula-2}
\end{equation}
где $r$ --- доходность от операций с акцией;\\
$P_s$ --- цена продажи акции;\\
$P_p$ --- цена покупки акции;\\
$\overline{Div}$ --- средний дивиденд за n лет (определяется как среднее арифметическое); 
$n$ --- число лет с момента покупки до момента продажи акции.

Пояснение каждого символа следует давать с новой строки в той же последовательности, что и в формуле. Первая строка пояснения должна начинаться со слова <<где>> без двоеточия после него.

Ссылки в тексте на порядковые номера формул дают в скобках, например, <<\dots{} в формуле \eqref{eq:formula-1}\dots>> или <<\dots{} в формуле \eqref{eq:formula-2}\dots>>.

Формулы, помещаемые в приложениях, должны нумероваться отдельно арабскими цифрами в пределах каждого приложения с добавлением перед каждым номером формулы обозначения (номера) приложения, например: <<\dots{} в формуле (B.1)\dots>> – ссылка на формулу номер 1 в приложении B.

\section{Оформление илюстраций}

Количество иллюстраций в работе должно быть достаточным для пояснения излагаемого текста. Иллюстрации (графики, диаграммы, скриншоты, блок-схемы и др.) могут быть расположены как по тексту работы (как можно ближе к соответствующим частям текста), так и в конце ее (в приложении). Текст на иллюстрациях должен быть читабельным, размер кегля не менее 7.

Иллюстрации должны быть выполнены в соответствии с требованиями стандартов ЕСКД и СПД (ГОСТ 2.004-88)\footnote{ГОСТ 2.004-88. Общие требования к выполнению конструкторских и технологических документов на печатающих и графических устройствах вывода ЭВМ. М., 1988. 27 с. (Единая система конструкторской документации).}.

На все размещенные в работе рисунки должны быть ссылки в ее тексте. Ссылка на рисунок задается в формате «рис. C.N». В качестве примера см.~рис.~\ref{fig:example-figure}.

\begin{FIGURE}[t]{Пример картинки\label{fig:example-figure}}
	\includegraphics[width=0.4\textwidth]{img/fig}
\end{FIGURE}

\section{Оформление списка литературы}

При использовании литературных источников, цитировании различных авторов, необходимо делать соответствующие ссылки, а в конце работы помещать список использованной литературы. Не только цитаты, но и произвольное изложение заимствованных из литературы принципиальных положений включаются в выпускную квалификационную работу со ссылкой на источник.

Список использованных источников приводится сразу за заключением. Для его оформления используется ГОСТ Р 7.0.5--20081. Ссылки также оформляются в соответствии с заданными правилами.

Библиографический список включает в себя литературные, статистические и другие источники, материалы которых использовались при написании ВКР.
Список состоит из таких литературных источников, как монографическая и учебная литература, периодическая литература (статьи из журналов и газет), законодательные и инструктивные материалы, статистические сборники и другие отчетные и учетные материалы, Интернет-сайты. Порядок построения списка определяется автором ВКР и руководителем.

В текст работы могут быть включены цитаты. При воспроизведении чужого текста ссылка на источник является необходимой, иначе возникают признаки плагиата – кражи интеллектуальной собственности. Во многих странах введены более строгие правила защиты интеллектуальной собственности, чем в России: требуется разрешение владельца авторского права даже на воспроизведение короткого фрагмента текста. В РФ можно цитировать чужой текст (обязательно 
со ссылкой) объемом до 300 знаков. В основном тексте работы должны присутствовать ссылки на все источники из библиографического списка.

Ссылки на источник размещаются в квадратных скобках внутри предложения. 
При цитировании текста цитата приводится в кавычках, а после нее в квадратных скобках указывается ссылка на литературный источник по списку использованной литературы и номер страницы, на которой в этом источнике помещен цитируемый текст. Если ссылка на источник приведена в конце предложения, то точка ставится после нее.

Ссылки на источник в \LaTeX{} даются командой \texttt{\string\cite} вне зависимости от типа источника. Информация об источниках должна быть размещена в \texttt{bibtex} файле, в данном случае в файле \texttt{library.bib}. Описание источника начинается с указания его типа \texttt{{@article}}, \texttt{{@book}}, \texttt{{@inproceedings}}, \texttt{{@online}} и~др., далее идёт описание специфичных полей для этого типа источника. Сам файле указывается в самом начале \LaTeX{} файла командой \texttt{\string\bibliography}. Для управления списком литературы рекомендуется использовать специализированные системы, например, \texttt{Mendeley}.

Команда \texttt{\string\cite} позволяет ссылаться на все типы источников по их имени: напимер, ссылка на книгу~\cite{BookExample}, или на сайт~\cite{HSEDocuments}. Также можно ссылаться сразу на несколько источников~\cite{ArticleExample,BookExample,ConferencePaperExample} или~\cite{HSEDocuments,ArticleExample,BookExample,ConferencePaperExample}. Обратите внимание, что в этом случае нельзя добавлять пробелы между именами источников.

Список литературы вставляется специальной командой, далее \LaTeX{} берёт на себя оформление списка литературы в соответствии с ГОСТом.


\putbibliography %Вместо этой команды будет вставлена библиография

\end{document}
