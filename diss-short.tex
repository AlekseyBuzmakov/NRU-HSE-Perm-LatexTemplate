\documentclass[PI,VKR]{HSEUniversity}
% Возможные опции: KR или VKR; PI или BI

\usepackage{lipsum} % Можно удалить, нужно только для команды \lipsum, которая вставляет бессмысленный текст.

\title{Краткий шаблон, демонстрирующий использование различных команд \LaTeX}
\author{Семён Семёныч Сидоров}
\supervisor{к.т.н., доцент кафедры Информационных технологий в бизнесе НИУ ВШЭ-Пермь}{И.И.~Иванов}
\reviewer{к.т.н., доцент кафедры Информационных технологий в бизнесе НИУ ВШЭ-Пермь}{П.П.~Петров}
\Year{2018}
\Abstract{
	После титульного листа размещается краткая (до 0,5 стр.) аннотация, предназначенная для реферативных изданий (например, журналы ВИНИТИ) и библиотечных информационных систем. В ней перечисляются автор, наименование работы; о чем она написана и для кого; количество страниц, иллюстраций, год, издательство (в данном случае – кафедра). Пример аннотации можно увидеть в любой книге на обороте титульного листа. Аннотации работ используются при формировании каталога работ, выполненных на кафедре. Текст аннотации оформляется в соответствии с правилами оформления основного текста работы.
}

% Ссылка на файл с описание библиографии
\bibliography{library.bib}

%%%%%%%%%%%%%%%%%%%%%%%%%%%%%%%%
%%% ТЕКСТ РАБОТЫ %%%%%%%%%%%%%%%
\begin{document}

% Обязательные элементы оформления: заголовочный слайд, аннотация, оглавление
\maketitle

\chapter*{Введение}

Слова в тексте могут быть выделены, например: \emph{Введение} представляет собой наиболее ответственную часть любой работы.
Также покажем как пользоваться <<кавычками>>. И <<Кавычками ``внутри'' кавычек>>

Далее будет продемонстрирована работа основных команд \LaTeX{}.

\chapter{Написание текста работы}

\section{Пример нумерованных списков}

Традиционно во введении:
\begin{enumerate}
	\item  обосновывается \emph{актуальность} выбранной темы;
	\item  формулируется \emph{цель работы} и \emph{содержание поставленных задач}, излагается их суть;
	\item  описываются \emph{объект} и \emph{предмет} исследования;
	\item  освещается \emph{степень разработанности} данной проблемы;
	\item  указывается направление и \emph{избранный метод (методы)} исследования, подходы к решению поставленных задач или реализации новой разработки;
	\item  указывается, что нового вносится автором в предмет исследования, отмечается \emph{теоретическая значимость} и \emph{прикладная ценность} планируемых результатов;
	\item  формулируются \emph{основные положения, которые автор выносит на защиту}.
\end{enumerate}

\section{Пример ненумерованных списков}

Традиционно во введении:
\begin{itemize}
	\item  обосновывается \emph{актуальность} выбранной темы;
	\item  формулируется \emph{цель работы} и \emph{содержание поставленных задач}, излагается их суть;
	\item  описываются \emph{объект} и \emph{предмет} исследования;
	\item  освещается \emph{степень разработанности} данной проблемы;
	\item  указывается направление и \emph{избранный метод (методы)} исследования, подходы к решению поставленных задач или реализации новой разработки;
	\item  указывается, что нового вносится автором в предмет исследования, отмечается \emph{теоретическая значимость} и \emph{прикладная ценность} планируемых результатов;
	\item  формулируются \emph{основные положения, которые автор выносит на защиту}.
\end{itemize}

\section{Заголовки разного уровня}

\lipsum[2]

Могут быть ещё подразделы

\subsection{Подраздел}

\lipsum[2]

И под-подразделы

\subsubsection{Под-подраздел}

\lipsum[2]


\section{Оформление таблиц} 
%
В качестве примера таблицы см.~табл.~\ref{tbl:example-table}.

\begin{TABLE}[h]{Пример таблицы\label{tbl:example-table}}
	\begin{tabular}{c|cc}
		\hline\hline % Двойная линия сверху таблицы
		Заголовок 1 & Заголовок 2 & Заголовок 3 \\ 
		\hline % Одинарная линия между заголовками и телом таблицы
		1 & 2 & 3 \\
		4 & 5 & 6 \\
		\hline\hline % Двойная линия снизу таблицы
	\end{tabular}
\end{TABLE}

\section{Оформление формул}

Для оформления формул используются стандартные средства \LaTeX{}. Примеры <<inline>> (внутри-строчных формул): $A^i_k$, $A_k^i$, $1+2+\cdots+n$, $x_1, x_2, \dots, x_n$.

Формулы, за исключением формул, помещаемых в приложении, должны нумероваться сквозной нумерацией арабскими цифрами, которые записывают на уровне формулы справа в круглых скобках, например:
\begin{equation}
	X^* = \frac{r_p}{\sqrt{(M-m_0\cdot I)V^{-1}(M-m_0\cdot I)}}\cdot V^{-1}(M-M_0\cdot I)
	\label{eq:formula-1}
\end{equation}

Пояснения символов и числовых коэффициентов, входящих в формулу, если они не пояснены ранее в тексте, должны быть приведены непосредственно под формулой:
\begin{equation}
	r = \frac{(P_s-P_p)/n + \overline{Div}}{(P_s + P_p)/2},
	\label{eq:formula-2}
\end{equation}
где $r$ --- доходность от операций с акцией;\\
$P_s$ --- цена продажи акции;\\
$P_p$ --- цена покупки акции;\\
$\overline{Div}$ --- средний дивиденд за n лет (определяется как среднее арифметическое); 
$n$ --- число лет с момента покупки до момента продажи акции.

Пояснение каждого символа следует давать с новой строки в той же последовательности, что и в формуле. Первая строка пояснения должна начинаться со слова <<где>> без двоеточия после него.

Ссылки в тексте на порядковые номера формул дают в скобках, например, <<\dots{} в формуле \eqref{eq:formula-1}\dots>> или <<\dots{} в формуле \eqref{eq:formula-2}\dots>>.

\section{Оформление илюстраций}

\begin{FIGURE}[t]{Пример картинки (вверх страницы)\label{fig:example-figure}}
	\includegraphics[width=0.4\textwidth]{img/fig}
\end{FIGURE}

На рис.~\ref{fig:example-figure} привидён пример картинки, которая располагается вверху страницы. На рис.~\ref{fig:example-figure-2} --- пример картинки, которая распалагается в том месте, в котором она была расположена.

\begin{FIGURE}[h]{Пример картинки (в месте расположения) \label{fig:example-figure-2}}
	\includegraphics[width=0.4\textwidth]{img/fig}
\end{FIGURE}


\section{Оформление списка литературы}

Ссылки на источник в \LaTeX{} даются командой \texttt{\string\cite} вне зависимости от типа источника. Информация об источниках должна быть размещена в \texttt{bibtex} файле, в данном случае в файле \texttt{library.bib}. Описание источника начинается с указания его типа \texttt{{@article}}, \texttt{{@book}}, \texttt{{@inproceedings}}, \texttt{{@online}} и~др., далее идёт описание специфичных полей для этого типа источника. Сам файле указывается в самом начале \LaTeX{} файла командой \texttt{\string\bibliography}. Для управления списком литературы рекомендуется использовать специализированные системы, например, \texttt{Mendeley}.

Команда \texttt{\string\cite} позволяет ссылаться на все типы источников по их имени: напимер, ссылка на книгу~\cite{BookExample}, или на сайт~\cite{HSEDocuments}. Также можно ссылаться сразу на несколько источников~\cite{ArticleExample,BookExample,ConferencePaperExample} или~\cite{HSEDocuments,ArticleExample,BookExample,ConferencePaperExample}. Обратите внимание, что в этом случае нельзя добавлять пробелы между именами источников.

Список литературы вставляется специальной командой \string\putbibliography, далее \LaTeX{} берёт на себя оформление списка литературы.

\putbibliography %Вместо этой команды будет вставлена библиография

\end{document}
